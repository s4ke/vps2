\documentclass[10pt,a4paper]{article} [2003/01/01]
\usepackage[german,ngerman]{babel}
\usepackage[utf8]{inputenc}
\usepackage[T1]{fontenc}
\usepackage{amssymb,amsmath,amsthm,pdfpages,graphicx,subfig,polynom} 
\usepackage{setspace}
\usepackage{lmodern, textcomp}
\usepackage{color}
\usepackage{wasysym}

% Diagrams
%\usepackage[all]{xy}
\usepackage{xyling}

\usepackage{listings}
\lstset{numbers=left, numberstyle=\tiny, numbersep=5pt}
\lstset{language=C++}

\usepackage{geometry}
\geometry{a4paper,left=15mm,right=20mm, top=1cm, bottom=2cm} 

\newcommand{\pvec}{\begin{pmatrix}}
\newcommand{\ppvec}{\end{pmatrix}}
\newcommand{\thus}{\ \Rightarrow\ }
\newcommand{\blank}{$\textblank$}

\newcommand{\sumxy}{\sum\limits_{x,y}}
\newcommand{\sumxys}{\sum\limits_{x',y'}}


\title{3. Übung Verteilte und Parallele Systeme 2}
\author{ Robert Günther (1145388) , Georg Rollinger (1161663) , Martin Braun(?) , 19.4.2013}
\date{}

\begin{document}
\maketitle
\begin{enumerate}

%Aufgaben in diesen Stil	
\item[]{\textbf{3.2}
\\Folgende Performance hat sich beim ausführen der ipsnd/iprcv Kombination ergeben:

256 byte Paketgröße:\\
	116kb:\\
	took 0.080696 seconds to receive 50953 bytes\\
	1.7mb:\\
	took 0.992676 seconds to receive 1006634 bytes\\
1024 byte Paketgröße:\\
	116kb:\\
	took 0.054936 seconds to receive 36340 bytes\\
	1.7mb:\\
	took 0.813740 seconds to receive 679347 bytes\\

Benutzt wurde ein Rechner aus dem Uni Netz zusammen mit einem vServer,
der bei Hetzner gehostet ist. Getestet wurde mit zwei unterschiedlichen
Dateigrößen die per cat | ./ipsnd <ip> <port> übertragen wurden. Man
sieht, dass es wichtig ist die richtige Paketgröße zu wählen, weil sonst
unnötig viele Bytes mitübertragen werden müssen (overhead des Protokolls).
Die Anzahl der Bytes von oben sind aufeinander-addierte Werte von read(...)
(Rückgabewert). Es wurde im Quelltext zudem nur die Zeit von vor read(...)
bis nach read(...) gemessen um den Output nicht mitzumessen.

}



\end{enumerate}
\end{document}
